% Created 2025-08-25 Mon 17:33
% Intended LaTeX compiler: pdflatex
\documentclass{beamer}\usepackage{listings}
\usepackage{color}
\usepackage{amsmath}
\usepackage{array}
\usepackage[T1]{fontenc}
\usepackage{natbib}
\lstset{
keywordstyle=\color{blue},
commentstyle=\color{red},stringstyle=\color[rgb]{0,.5,0},
basicstyle=\ttfamily\small,
columns=fullflexible,
breaklines=true,
breakatwhitespace=false,
numbers=left,
numberstyle=\ttfamily\tiny\color{gray},
stepnumber=1,
numbersep=10pt,
backgroundcolor=\color{white},
tabsize=4,
literate={~}{$\sim$}{1}{ø}{{\o}}1{æ}{{\ae}}1{å}{{\aa}}1{Ø}{{\OE}}1{Æ}{{\AE}}1{Å}{{\AA}}1,keepspaces=true,
showspaces=false,
showstringspaces=false,
xleftmargin=.23in,
frame=single,
basewidth={0.5em,0.4em},
}
\institute{}
\subtitle{for the prediction modeling strategy in general and the estimation of prediction performance in particular}
\RequirePackage[absolute, overlay]{textpos}
\setbeamertemplate{footline}[frame number]
\setbeamertemplate{navigation symbols}{}
\RequirePackage{fancyvrb}
\RequirePackage{array}
\RequirePackage{multirow}
\RequirePackage{tcolorbox}
\definecolor{mygray}{rgb}{.95, 0.95, 0.95}
\newcommand{\mybox}[1]{\vspace{.5em}\begin{tcolorbox}[boxrule=0pt,colback=mygray] #1 \end{tcolorbox}}
\newcommand{\sfootnote}[1]{\renewcommand{\thefootnote}{\fnsymbol{footnote}}\footnote{#1}\setcounter{footnote}{0}\renewcommand{\thefootnote}{\arabic{foot note}}}
\usepackage{alphalph}
\alphalph{\value{footnote}}
\setbeamertemplate{itemize item}{\textbullet}
\setbeamertemplate{itemize subitem}{-}
\setbeamertemplate{itemize subsubitem}{}
\setbeamertemplate{enumerate item}{\insertenumlabel.}
\setbeamertemplate{enumerate subitem}{\insertenumlabel.\insertsubenumlabel}
\setbeamertemplate{enumerate subsubitem}{\insertenumlabel.\insertsubenumlabel.\insertsubsubenumlabel}
\setbeamertemplate{enumerate mini template}{\insertenumlabel}
\makeatletter\def\blfootnote{\xdef\@thefnmark{}\@footnotetext}\makeatother
\newcommand{\E}{\ensuremath{\mathrm{E}}}
\renewcommand{\P}{\ensuremath{\mathrm{P}}}
\renewcommand{\d}{\ensuremath{\mathrm{d}}}
\setbeamertemplate{caption}{\raggedright\insertcaption\par}

\renewcommand*\familydefault{\sfdefault}
\itemsep2pt
\usepackage[utf8]{inputenc}
\usepackage[T1]{fontenc}
\usepackage{graphicx}
\usepackage{longtable}
\usepackage{wrapfig}
\usepackage{rotating}
\usepackage[normalem]{ulem}
\usepackage{amsmath}
\usepackage{amssymb}
\usepackage{capt-of}
\usepackage{hyperref}
\usetheme{default}
\date{}
\title{Obstacles and pitfalls}
\begin{document}

\maketitle
\section{Modeling}
\label{sec:orge8f0ab9}

\begin{frame}[label={sec:orgebec7b2}]{Outline}
Advanced statistical topics arise routinely in the prediction
modeling process and require special attention.
\vfill
Many issues
have no clear solutions but they are still important to
recognize.
\vfill
However, even unsolved, these issues are important to point
out when discussing the limitations of a study, and potentially could
motivate the target of a sensitivity analysis.
\vfill

Cross-validation, Censored outcome, Missing predictor values
\end{frame}
\setbeamercolor{background canvas}{bg=black}
\begin{frame}[label={sec:org4086a70}]{}
\huge \color{white}
Cross-validation
\end{frame}
\setbeamercolor{background canvas}{bg=white}
\begin{frame}[label={sec:org8bf1545}]{Data splitting}
Internal cross-validation can be used to tune model parameters and to
assess and compare the predictive performance of a list of candidate
models.

\vfill The term ``internal'' can either refer to the fact that the
``validation'' study is performed by the same team who made the model,
or it can refer to the fact that the ``validation'' study splits a
purpose data into training and validation sets \vfill

We generally \alert{discourage} the use of a \alert{single split} of data into one set
for training and one for validation.
\end{frame}
\begin{frame}[label={sec:orgd4bedad}]{Data splitting}
\begin{center}
\includegraphics[width=.9\linewidth]{/home/tag/metropolis/Teaching/causal-prediction-workshop/lecturenotes/figure-7.1.pdf}
\end{center}
\end{frame}
\begin{frame}[label={sec:orgf9a7ba8}]{Data splitting}
\begin{center}
\includegraphics[width=.9\linewidth]{/home/tag/metropolis/Teaching/causal-prediction-workshop/lecturenotes/figure-7.3.pdf}
\end{center}
\end{frame}
\begin{frame}[label={sec:org11a0a92}]{Learning curve}
\vspace{-3em}

\begin{center}
\includegraphics[width=.9\linewidth]{/home/tag/metropolis/Teaching/causal-prediction-workshop/lecturenotes/figure-7-learning-curve.pdf}
\end{center}
\end{frame}
\begin{frame}[label={sec:org208a389},fragile]{Example: single split results depend on seed}
 \begin{lstlisting}[language=r,numbers=none,otherkeywords={seed,train,method,B,M}, deletekeywords={model,family,se,null,data,formula}]
x <- Score(list("Conventional model" = conventional_model,"Experimental model" = experimental_model),
           data = train,
           formula = cvd_5year~1,
           split.method = "bootcv",
           progress.bar = NULL, verbose = FALSE,
           null.model = FALSE, se = FALSE,
           seed = 17,
           B = 1,
           M = .632*NROW(train))
summary(x,what = "score")
\end{lstlisting}

\phantomsection
\label{}
\begin{verbatim}
$score
Key: <Model>
                Model AUC (%) Brier (%)
               <fctr>  <char>    <char>
1: Conventional model    91.6       5.7
2: Experimental model    90.3       5.9
\end{verbatim}
\end{frame}
\begin{frame}[label={sec:orgdd6680a},fragile]{Example: single split results depend on seed}
 \begin{lstlisting}[language=r,numbers=none,otherkeywords={seed,train,method,B,M}, deletekeywords={model,family,se,null,data,formula}]
x <- Score(list("Conventional model" = conventional_model,"Experimental model" = experimental_model),
           data = train,
           formula = cvd_5year~1,
           progress.bar = NULL, verbose = FALSE,
           null.model = FALSE, se = FALSE,
           split.method = "bootcv",
           seed = 4,
           B = 1,
           M = .632*NROW(train))
summary(x,what = "score")
\end{lstlisting}
\end{frame}
\begin{frame}[label={sec:org2f088c1},shrink=25]{Repeated cross-validation helps (but changes the target!)}
\begin{center}
\includegraphics[width=.9\linewidth]{/home/tag/research/Methods/PredictionModelsMonograph/figure-7.88-colour.jpg}
\end{center}
\end{frame}
\begin{frame}[label={sec:org7716257},fragile]{Repeated cross-validation helps (but changes the target!)}
 \begin{lstlisting}[language=r,numbers=none,otherkeywords={seed,train,method,B,M}, deletekeywords={model,family,se,null,data,formula}]
x <- Score(list("Conventional model" = conventional_model,"Experimental model" = experimental_model),
           data = train,
           formula = cvd_5year~1,
           progress.bar = NULL, verbose = FALSE,
           null.model = FALSE, se = FALSE,
           split.method = "bootcv",
           seed = 17,
           B = 100,
           M = .632*NROW(train))
summary(x,what = "score",digits = 2)
\end{lstlisting}

\phantomsection
\label{}
\begin{verbatim}
$score
Key: <Model>
                Model AUC (%) Brier (%)
               <fctr>  <char>    <char>
1: Conventional model   90.95      4.98
2: Experimental model   90.62      5.11
\end{verbatim}
\end{frame}
\begin{frame}[label={sec:org211b5d8},fragile]{Repeated cross-validation helps (but changes the target!)}
 \begin{lstlisting}[language=r,numbers=none,otherkeywords={seed,train,method,B,M}, deletekeywords={model,family,se,null,data,formula}]
x <- Score(list("Conventional model" = conventional_model,"Experimental model" = experimental_model),
           data = train,
           formula = cvd_5year~1,
           progress.bar = NULL, verbose = FALSE,
           null.model = FALSE, se = FALSE,
           split.method = "bootcv",
           seed = 4,
           B = 100,
           M = .632*NROW(train))
summary(x,what = "score",digits = 2)
\end{lstlisting}

\phantomsection
\label{}
\begin{verbatim}
$score
Key: <Model>
                Model AUC (%) Brier (%)
               <fctr>  <char>    <char>
1: Conventional model   90.88      4.98
2: Experimental model   90.60      5.11
\end{verbatim}
\end{frame}
\setbeamercolor{background canvas}{bg=black}
\begin{frame}[label={sec:org3e741de}]{}
\huge \color{white}
Censored time to event data
\end{frame}
\setbeamercolor{background canvas}{bg=white}
\begin{frame}[label={sec:org0c59a7c}]{Censored data}
A common behavior is that patients with short follow-up are excluded
from the dataset. This is almost never the right thing to do, as
intuitive as it may seem. Excluding them results in bias.
\vfill

Outcome at the prediction time horizon is unknown (censored) for
subjects who were not followed until the prediction time horizon and
were free of any event by the end of their individual follow-up
period.  \vfill

Subjects who die for other reasons before the time horizon are not censored!

\vfill
Specific estimation techniques are needed to take care of bias due to
censored data.
\end{frame}
\begin{frame}[label={sec:org2acabac},shrink=25]{}
\begin{center}
\includegraphics[width=.9\linewidth]{/home/tag/metropolis/Teaching/causal-prediction-workshop/lecturenotes/figure-2-time2event-data.pdf}
\end{center}

At the 5-year prediction time horizon, only patient 5 is censored. At
the 10-year prediction time horizon, patients 5 and 6 are censored.
\end{frame}
\begin{frame}[label={sec:orgd72a7f1}]{Inverse probability of censoring weighting}
To estimate the average Brier score we calculate a weighted average
for only those subjects for who the event status
at the prediction time horizon is known (groups \emph{event} and
\emph{event-free}).

\vfill The weight of a subject is higher the more (similar) subjects
were lost to follow-up (\emph{censored}) earlier in time.

\vfill This is to account for the fact that those in the \emph{censored}
group could possibly have experienced the event by the prediction time
horizon, just this is not observed.

\vfill
Subjects in the \emph{censored} group are not directly used in
the weighted average. However, they enter indirectly through the
weights assigned to other subjects.  
\end{frame}
\begin{frame}[label={sec:orgad11501}]{Choice of the prediction time horizon}
The prediction time horizon should be chosen such that it is of
subject matter interest to predict the probability that the event
occurs in the time period between the time origin and the prediction
time horizon.
\vfill

The shorter the prediction time horizon the fewer subjects are
censored!
\vfill

It is sometimes useful to study multiple
prediction time horizons, but this should be motivated by the
application at hand (beware of cherry-picking).
\end{frame}
\begin{frame}[label={sec:orgab6cb05},fragile]{Example code (does not work with course data)}
 \begin{lstlisting}[language=r,numbers=none,otherkeywords={formula,times,event,Hist}, deletekeywords={model,data,NROW,split,null}]
x <- Score(list("Conventional model" = conventional_model,"Experimental model" = experimental_model),
           data = train,
           formula = Hist(time,event)~age+sex+...,
           times = 1:10,
           split.method = "cv10",
           seed = 4,
           B = 100,
           M = .632*NROW(train))
\end{lstlisting}
\end{frame}
\setbeamercolor{background canvas}{bg=black}
\begin{frame}[label={sec:org9d8d1d7}]{}
\huge \color{white}
Missing predictor values
\end{frame}
\setbeamercolor{background canvas}{bg=white}
\begin{frame}[label={sec:org513932a}]{Missing predictor values}
When the target of the analysis is an association parameter such as an
odds ratio or a hazard ratio, then \emph{multiple imputation} or \emph{inverse
probability weighting} may increase power compared to a \emph{complete case
analysis}.
\vfill
However, it is not so simple that multiple imputation has always less bias 
than a complete case analysis.
\vfill
When the aim is to build and ``validate'' a risk prediction model, we need to
deal with missing values in two conceptually different places: 

\begin{itemize}
\item missing values in the learning dataset
\item missing values in data of a new patient who is asking for a predicted risk
\end{itemize}
\end{frame}
\begin{frame}[label={sec:orgfdca57a}]{Reason for missing predictor values}
We need to understand \alert{why} the values are missing and explore the reasons.

\begin{center}
\begin{tabular}{m{3cm}m{8cm}}
\hline
\emph{Missing completely at random} & The probability of a missing value is not related to either outcome or predictor variables.\\
\hline
\emph{Missing at random} & The probability of a missing value is predictable by the other predictor variables \alert{and outcome}.\\
\hline
\emph{Missing not at random} & The probability of a missing value depends on the missing value and/or on other unobserved variables.\\
\hline
\end{tabular}
\end{center}

An inconvenient truth to realize is that the type of missingness can
generally not be inferred from the data. 
\end{frame}
\begin{frame}[label={sec:org8e5d03e}]{Learning phase (Making of a prediction model)}
Missing values in the predictor variables affect the choice of
predictor variables: having few predictor variables with a high percentage of
missingness can be equally bad as having many predictor variables with few missing values in different subjects.
\vfill

Any modelling algorithm (logistic regression, random forest, etc.) may
use missingness of a variable as a predictor and also use imputation
of missing values as part of the algorithm!

\vfill

Imputation means to replace a missing value with a likely value.  For
example, to impute a value for BMI we could specify a linear
regression model which relates BMI to the other predictor variables,
other auxiliary variables, and the outcome.
\end{frame}
\begin{frame}[label={sec:orgf8d11ec}]{Imputation algorithms should also condition on outcome}
Thinking about how missingness can possibly depend on an outcome
variable is generally cumbersome, and it does not get any easier
when the outcome is a right-censored time-to-event variable.
\vfill

If the missingness of a predictor variable "happens" at baseline, then
there cannot be a direct causal effect of the outcome variable on the
missingness. 
\vfill

Hence, if the missingness depends on a time-to-event
variable conditional on the observed predictor variables, this must
mean that there exists an unobserved variable, such as disease burdon,
which mediates the relationship.
\end{frame}
\begin{frame}[label={sec:orgf423c20},shrink=25]{}
\begin{figure}[htbp]
\centering
\includegraphics[width=0.7\textwidth]{/home/tag/metropolis/Teaching/causal-prediction-workshop/lecturenotes/figure-7.35.pdf}
\caption{Suppose the substantive model is a Cox regression model which shows that the 5-year survival probability increases with increasing BMI. Without conditioning on the outcome, we simply impute missing BMI values for every one from a normal distribution with mean given by the mean BMI in the complete cases. When the 5-year survival probability is an increasing function of BMI as in the figure this imputed value (the black dot on the line) is systematically too large for subjects with BMI below 25 and systematically too low for subjects with BMI above 25.}
\end{figure}
\end{frame}
\begin{frame}[label={sec:orgcb84473}]{Missing values in the validation data}
We distinguish the following two tasks.
\vfill
The first task is
to have the estimator of prediction performance deal with missing values in the
validation dataset.

\vfill

The second task is to enrich the model such that
the input values of some of the predictor variables become optional
for the user of the model.
\vfill
That is, if the user of the model cannot provide the value of one or several predictors, ideally, then the
model should still be able to provide a predicted risk. 
\vfill

Obviously, all this makes most sense when the missingness in the
validation dataset resembles the expected missingness in the future
users of the model.
\end{frame}
\end{document}
